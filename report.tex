% !TEX TS-program = xelatex
%
\documentclass{report}

\usepackage[english]{babel}
\usepackage{parskip}
\title{CG-Report}
\author{Markus Reiter, Hüseyin Gündogan, Michael Kaltschmid}

\begin{document}
\maketitle

The goal of the proseminar computer graphics this year was to create a merry-go-round in OpenGL. There were 5 assignments to complete throughout the whole lecture. With each assignment the complexity of the carousel increased further. 
\par
the first assignment was about setting up the basic geometrical model and adding a basic animation. the second assignment asked for improvements concerning the geometrical model and the addition of camera controls including a camera which moves around the base in an automatic fashion. Some more advanced features like controllable light sources and phong shading were required for the completion of the third assignment. The fourth task was all about texturing and lastly it was up to the students to come up with an unique effect which enhances the merry-go-round and finalizes the project.
\par
Our team consisted out of three members which are Markus Reiter, Huesyin Gündogan and Michael Kaltschmid. The organization was straight forward. Pretty much all of our communication and decisions regarding the project were made online. For structuring our code we decided on using GitHub. GitHub offers a very clean user interface with rich functionality and therefore was a great tool for keeping all our progress synchronized. 
\par
Our merry-go-round offers all the basic geometry structures which are to be expected. To make our lives easier we created functions which were of great help in the process of creating abstract shapes for the carousel. cubes, cylinders, rectangles and variations of those shapes were among a vast array of forms which ended up being the basis of many creations which followed. 
\par
We have implemented a night mode as well. It is a neat little feature where the colour of the background changes according to the time of day.
\par
The basis for shading in our project are still fragment\_shader and vertex\_shader although heavily modified versions to support phong shading, fog and lightning. For lightning in particular we also have a helper shader called light\_shader. To ensure better structure the billboard in our scene has its own implementation of the fragment\_shader and vertex\_shader.
\par
The merry-go-round offers also an extensive range of controls were lightning, hue and specular can be changed. In addition to that we offer a variety of camera controls. Moving the merry-go-round around the scene and changing the rotation are among the most prominent features in that category.
\par
To enhance the carousel further we incorporated objects in the form of planes and a palm tree which are loaded through a obj loader. To diversify the scene and to give it a unique touch, textures were added. Textures are loaded through a basic interfaces which applies images to objects of the merry-go-round.
\pagebreak
Ultimately we decided on implementing the effect fog which not only enhances the merry-go-round and the whole scene but also serves a purpose. We still had other fitting effects in mind but decided against them mostly because of time limitations and lack of effort that would be required by some of the more advanced effects like Ambient Occlusion or reflections.  
\par
In the end it is safe to say that we all enjoyed the experience that was implementing a merry-go-round in OpenGL. OpenGL proved to be difficult at first. Especially when it comes to trying to get it to work on windows but as time progressed and we all had spend more time with the tool we came to the realization that it is very powerful and functional.


\end{document}