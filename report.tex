% !TEX TS-program = xelatex
%
\documentclass{report}

\usepackage[english]{babel}
\usepackage{parskip}
\title{CG-Report}
\author{Markus Reiter, Hüseyin Gündogan, Michael Kaltschmid}

\begin{document}
\maketitle

The goal of the proseminar computer graphics this year was to create a merry-go-round in OpenGL. There were 5 assignments to complete throughout the whole lecture. With each assignment the complexity of the carousel increased further. 
\par
the first assignment was about setting up the basic geometrical model and adding a basic animation. the second assignment asked for improvements concerning the geometrical model and the addition of camera controls including a camera which moves around the base in an automatic fashion. Some more advanced features like controllable light sources and phong shading were required for the completion of the third assignment. The fourth task was all about texturing and lastly it was up to the students to come up with an unique effect which enhances the merry-go-round and finalizes the project.
\par
Our team consisted out of three members which are Markus Reiter, Huesyin Gündogan and Michael Kaltschmid. The organization was straight forward. Pretty much all of our communication and decisions regarding the project were made online through the text messaging app Telegram. For structuring our code we decided on using GitHub. GitHub offers a very clean and functional user interface and rich functionality and therefore was a great tool for keeping all our progress synchronized. 
\par
Most of our time during the first assignment was devoted to building function which were of great help in the process of creating abstract shapes for the carousel. cubes, cylinders and variations of those shapes were among a vast array of forms which ended up being the basis of many creations which followed. To make our lives easier we also introduced various macros. One of our favourites has to be the notorious \texttt{RGB\_RAND} macro which assigned random colours to our objects. Getting a grip of how OpenGL works was also a major aspect in the process of completing the first task. We were able to meet all requirements and we were proud of the result especially with the already mentioned helper function which paved the way for all assignments to follow.
\par
Our first goal for the second assignment was to refactor most of the project up until this date. Switching to C++ was also a discussion point which would have possibly lead to further modularity but was quickly dismissed because OpenGL in its core is a C API and therefore the introduction of C++ would not have been an improvement in every aspect. A proper C++ implementation would have also required a serious rewrite of many functions. Our main focus was still shifted towards implementing proper camera controls. Completing that task proved to be of not much difficulty. Making those controls work together with the automatic camera was a different story. Ultimately we came up with a satisfactory solution. Our beloved abstract shape functions came in handy ones more for the requirement of enhancing the geometry of the merry-go-round. In hindsight we probably should have added more enhancements to the geometry to completely fulfill the requirements of the task.
\par
We started off working on the third assignment as usual with a obligatory refactor. Before things overall went very smoothly but this time around we saw some flaws in our implementation and therefore had to rework some functions to accustom for the tasks of the assignment which among other were to work with shading in particular phong shading. In order to shade properly vertices had to be duplicated which was not possible with the up until this point implemented algorithm for the creation of objects. Things got also more difficult when working with lights as we had to smarten up on the OpenGL front. Both shading and lighting required to extend the fragment and light shader which were previously almost untouched. After some research and collaboration of the team members we managed to come up with a solution which not only met the requirements but also exceeded our expectations. Ultimately we decided on using two different light sources which served different purposes. one source lighted the scene from up top kind of like a sun and the other one lightened the interiors of the merry-go-round. Implementing the functionality of changing the HSV of the lights was fairly easy since we could reuse the event handlers from the keyboard controls of assignment 2.
\par
Things started to get a little dicey for various reasons during assignment 4. Mainly because  every team member had other time consuming obligations as well and therefore could not completely focus on the task for the better part of the time frame between the previous assignment and the deadline for the current assignment. Nevertheless we still managed to finish the task last minute. Texture loading and texturing various objects worked as expected right off the bad. The whole scene benefitted greatly by the added textures. The billboard on the other hand still left room for improvements.  We attribute this shortcoming to the already mentioned time constraints. Yet it still serves its purpose and for the most part works as intended. 
\par
Which leads us to the fifth and final assignment where nothing more than adding one special effect was required. We all had different ideas when it came to what this effect should be. Ultimately we decided on implementing an effect which not only enhances the merry-go-round and the whole scene but also is functional and serves a purpose. We still had other fitting effects in mind but decided against them mostly because of time limitations and lack of effort that would be required by some of the more advanced effects like Ambient Occlusion or reflections. 
\par
In the end it is safe to say that we all enjoyed the experience that was implementing a merry-go-round in OpenGL. OpenGL proved to be difficult at first. Especially when it comes to trying to get it to work on windows. As time progressed and we all spent more time with the tool we came to the realization that it is very powerful and functional. We also attribute a lot of our success to great teamwork and organization when it comes to the usage of git.  


\end{document}